\documentclass[11pt,a4paper]{article}

\usepackage{t1enc}
\usepackage[latin1]{inputenc}
\usepackage[english]{babel}
\usepackage[dvips]{hyperref}
\usepackage{amsmath,amssymb}
\usepackage{graphicx}
\usepackage{url}
\usepackage{float}

\usepackage[round]{natbib}
\bibliographystyle{jrss}

\pagestyle{plain}
\setlength{\parindent}{0in}
\setlength{\parskip}{1.5ex plus 0.5ex minus 0.5ex}
\setlength{\oddsidemargin}{0in}
\setlength{\evensidemargin}{0in}
\setlength{\topmargin}{-0.5in}
\setlength{\textwidth}{6.3in}
\setlength{\textheight}{9.8in}

\numberwithin{equation}{section}
\newcommand\email{\begingroup \urlstyle{rm}\Url}

\begin{document}

\begin{center}
  \LARGE 
  A User's Guide to the POT Package (Version 1.0) \\
  \Large
  \vspace{0.2cm}
  Mathieu Ribatet \\
  \normalsize
  Copyright \copyright 2006 \\
  \vspace{0.2cm}
  Department of Hydrological Statistic, INRS,\\
  University of Qu\'ebec, 490 de la Couronne, G1K 9A9 Canada\\
  \vspace{0.2cm}
  Cemagref UR HH, 3bis quai Chauveau\\ 69336 Lyon Cedex
  09 France\\ 
  \vspace{0.2cm}
  E-mail: \email{ribatet@hotmail.com} \\
  7th August 2006 
\end{center}

\section{Introduction}
\label{sec:intro}

\subsection{Why the POT package?}
\label{subsec:why}

The \textbf{POT} package is an add-on package for the R statistical
software~\citep{Rsoft}. The main goal of this package is to develop
tools to perform stastical analyses of Peaks Over a Threshold
(\textbf{POT}).

Most of functions are related to the Extreme Value Theory
(\textbf{EVT}). \citet{Coles2001} gives a comprehensive introduction
to the EVT, while~\citet{Kluppelberg1997} present advanced results.

\subsection{Obtaining the package/guide}

The package can be downloaded from CRAN (The Comprehensive R Archive
Network) at \url{http://cran.r-project.org/}.  This guide (in pdf)
will be in the directory \verb+POT/doc/+ underneath wherever the
package is installed.

\subsection{Contents}
\label{subsec:contents}

To help users to use properly the \textbf{POT} package, this guide
contains practical examples on the use of this
package. Section~\ref{sec:introEVT} introduce quickly the Extreme
Value Theory (\textbf{EVT}). Some basic examples are described in
section~\ref{sec:BasicUse}, while section~\ref{sec:concAn} gives a
concrete statistical analysis of extreme value for river Adie\'eres at
Beaujeu (FRANCE).

\subsection{Citing the package/guide}

To cite this guide or the package in publications please use the
following bibliographic database entry.
\begin{verbatim}
@Manual{key,
  title = {A User's Guide to the POT Package (Version 1.0)},
  author = {Ribatet, M. A.},
  year = {2006},
  month = {August},
  url = {http://cran.r-project.org/}
}
\end{verbatim}

\subsection{Caveat}

I have checked these functions as best I can but, as ever, they may
contain bugs.  If you find a bug or suspected bug in the code or the
documentation please report it to me at \verb+ribatet@hotmail.com+.
Please include an appropriate subject line.

\subsection{Legalese}

This program is free software; you can redistribute it and/or modify
it under the terms of the GNU General Public License as published by
the Free Software Foundation; either version 2 of the License, or (at
your option) any later version.

This program is distributed in the hope that it will be useful, but
without any warranty; without even the implied warranty of
merchantability or fitness for a particular purpose.  See the GNU
General Public License for more details.

A copy of the GNU General Public License can be obtained from
\url{http://www.gnu.org/copyleft/gpl.html}.  You can also obtain it by
writing to the Free Software Foundation, Inc., 59 Temple Place --
Suite 330, Boston, MA 02111-1307, USA.

\section{An Introduction the EVT}
\label{sec:introEVT}

Even if this package is only related to peaks over a threshold, a
classical introduction to the EVT must deal with ``block maxima''. Let
$X_1,\ldots,X_n$ be a series of independent and identically distributed
random variables with commom distribution function $F$. Let $M_n =
\max(X_1,\ldots,X_n)$.

Suppose there exists normalizing constants $a_n>0$ and $b_n$ such
that:
\begin{equation}
  \label{eq:convMax}
  \Pr\left[\frac{M_n - b_n}{a_n} \leq y \right] = F^n(a_ny+b_n)
  \longrightarrow G(y), \qquad n \rightarrow +\infty
\end{equation}
for all $y \in \mathbb{R}$, where $G$ is a non-degenerate distribution
function. According to the Extremal Types Theorem~\citep{Fisher1928},
$G$ must be either Fr\'echet, Gumbel or negative
Weibull. \citet{Jenkinson1955} noted that these three distributions can
be merged into a single parametric family: the Generalized Extreme
Value (\textbf{GEV}) distribution. The GEV has a distribution function
defined by:
\begin{equation}
  \label{eq:GEV}
  G(y) = \exp\left[ -\left( 1 + \xi \frac{y-\mu}{\sigma} \right)_+^{-1/\xi} \right], 
\end{equation}
where $(\mu,\sigma,\xi)$ are the location, scale and shape parameters
respectively, $\sigma>0$ and $z_+=\max(z,0)$.

The Fr\'echet case is obtained when $\xi>0$, the negative Weibull when
$\xi<0$ while the Gumbel case is defined by continuity when
$\xi\rightarrow0$.

From this result, \citet{Pickands1975} showed that the limiting
distribution of normalized excesses of a threshold $\mu$ as the
threshold approaches the endpoint $\mu_\mathrm{end}$ of the variable
of interest is the Generalized Pareto Distribution
(\textbf{GPD}). That is, if $X$ is a random variable which
holds~\eqref{eq:convMax}, then:
\begin{equation}
  \label{eq:convExcess}
  \Pr\left[X \leq y | X > \mu \right] \longrightarrow H(y), \qquad \mu
    \rightarrow \mu_\mathrm{end}
\end{equation}
with
\begin{equation}
  \label{eq:GPD}
  H(y) = 1 - \left(1 + \xi\frac{y-\mu}{\sigma}\right)_+^{-1/\xi},
\end{equation}
where $(\mu,\sigma,\xi)$ are the location, scale and shape parameters
respectively, $\sigma>0$ and $z_+=\max(z,0)$. Note that the
Exponential distribution is obtained by continuity as $\xi \rightarrow
0$.

In practice, these two asymptotical results motivated modelling block
maxima with a GEV, while peaks over threshold with a GPD.

\section{Basic Use}
\label{sec:BasicUse}

\subsection{Random Numbers and Distribution Functions}
\label{subsec:randDist}

First of all, lets start with basic stuffs. The \textbf{POT} package
uses the R convention for random numbers generation and distribution
function features. 
\begin{verbatim}
> ##Random number generation
> rgpd(5, loc = 1, scale = 2, shape = -0.2)
[1] 4.672547 2.365295 1.899087 1.577886 2.409450
> ##Varying threshold can be performed also
> rgpd(6, c(1, -5), 2, -0.2)
[1]  2.424368 -3.389774  3.965086 -3.332016  4.707819 -4.985408
> ##The same but with a varying scale parameter
> rgpd(6, 0, c(2, 3), 0)
[1] 2.9850740 3.1486256 1.0705649 0.7401753 3.1231517 2.3994109
> ##Probability of non exceedence
> pgpd(c(9, 15, 20), 1, 2, 0.25)
[1] 0.9375000 0.9825149 0.9922927
> ##Quantile associated to probability of non exceedence
> qgpd(c(.25, .5, .75), 1, 2, 0)
[1] 1.575364 2.386294 3.772589
> ##Evaluate the density at point...
> dgpd(c(9, 15, 20), 1, 2, 0.25)
[1] 0.015625000 0.003179117 0.001141829
\end{verbatim}
Several options can be passed to three of these four functions. In
particular:
\begin{itemize}
\item for ``pgpd'', user can specify if non exceedence or exceedence
  probability should be computed with option \verb+lower.tail = TRUE+
  or  \verb+lower.tail = FALSE+ respectively;
\item for ``qgpd'', user can specify if quantile is related to non
  exceedence or exceedence probability with option
  \verb+lower.tail = TRUE+ or \verb+lower.tail = FALSE+ respectively;
\item for ``dgpd'', user can specify if the density or the log-density
  should be computed with option \verb+log = FALSE+ or
  \verb+log = TRUE+ respectively.
\end{itemize}

\subsection{Threshold Selection}
\label{subsec:threshSelect}

The location for the GPD or equivalently the threshold is a particular
parameter as must often it is not estimated as the other ones. All
methods to define a suitable threshold use the asymptotic
approximation defined by equation~\eqref{eq:convExcess}. In other
words, we select a threshold for which the asymptotic distribution $H$
in equation~\eqref{eq:GPD} is a good approximation.

The \textbf{POT} package has several tools to define a reasonable
threshold. For this purpose, the user must use
\verb+tcplot, mrlplot, lmomplot, exiplot+ and \verb+diplot+ functions.

The main goal of threshold selection is to selects enough events to
reduce the variance; but not too much as we could select events coming
from the central part of the distribution\footnote{i.e. not extreme
  events.} and induce bias.

\subsubsection{Threshold Choice plot:  \emph{tcplot}}
\label{subsubsection:tcplot}

Let $X \sim GP(\mu_0, \sigma_0, \xi_0)$. Let $\mu_1$ be a another
threshold as $\mu_1 > \mu_0$. The random variable $X | X > \mu_1$ is
also GPD with updated parameters $\sigma_1 = \sigma_0 + \xi_0 ( \mu_1
- \mu_0)$ and $\xi_1 = \xi_0$.
Let
\begin{equation}
  \label{eq:reparamScale}
  \sigma_* = \sigma_1 - \xi_1 \mu_1
\end{equation}
With this new parametrization, $\sigma_*$ is independent of
$\mu_1$. Thus, estimates of $\sigma_*$ and $\xi_1$ are constant for
all $\mu_1 > \mu_0$ if $\mu_0$ is a suitable threshold for the
asymptotic approximation.

Threshold choice plots represent the points defined by:
\begin{equation}
  \label{eq:tcplot}
  \left\{\left(\mu_1, \sigma_*\right) : \mu_1 \leq x_\mathrm{max}
  \right\} \quad \text{and} \quad \left\{\left(\mu_1, \xi_1\right) :
    \mu_1 \leq x_\mathrm{max} \right\}
\end{equation}
where $x_\mathrm{max}$ is the maximum of the observations
$\mathbf{x}$.

Moreover, confidence intervals can be computed using Fisher
information. 

Here is an application.
\begin{verbatim}
> x <- runif(10000)
> par(mfrow=c(1,2))
> tcplot(x, u.range = c(0.9, 0.995))
\end{verbatim}
Results of the \verb+tcplot+ function is displayed in
Figure~\ref{fig:tcplot}. We can see clearly that a threshold around
0.98 is a reasonable choice. However, in practice decision are not so
clear-cut as for this synthetic example.
\begin{figure}
  \centering
  \includegraphics[angle=-90,width=1\textwidth]{tcplot}
  \caption{Threshold Choice plot on synthetic data}
  \label{fig:tcplot}
\end{figure}

\subsubsection{Mean Residual Life Plot: \emph{mrlplot}}

The \textbf{mean residual life plot} is based on the theoretical mean
of the GPD. Let $X$ be a \textit{r.v.} distributed as $GPD(\mu,
\sigma, \xi)$. Then, theoretically we have:
\begin{equation}
  \label{eq:meanGPD}
  \mathbb{E}\left[X \right] = \mu + \frac{\sigma}{1-\xi}, \qquad
  \text{for } \xi < 1
\end{equation}
When $\xi\geq1$, the theoretical mean is infinite.

In practice, if $X$ represents excess over a threshold $\mu_0$, and if
the approximation by a GPD is good enough, we have:
\begin{equation}
  \label{eq:meanExcess}
  \mathbb{E}\left[X - \mu_0 | X > \mu_0 \right] =
  \frac{\sigma_{\mu_0}}{1 - \xi}
\end{equation}
For all new threshold $\mu_1$ such as $\mu_1 > \mu_0$, excesses above
the new threshold are also approximate by a GPD with updated
parameters - see section~\ref{subsubsection:tcplot}. Thus,
\begin{equation}
  \label{eq:meanExcess2}
  \mathbb{E}\left[X - \mu_1 | X > \mu_1 \right] =
  \frac{\sigma_{\mu_1}}{1 - \xi} = \frac{\sigma_{\mu_0} +
    \xi \mu_1}{1 - \xi}
\end{equation}
The quantity $\mathbb{E}\left[X - \mu_1 | X > \mu_1 \right]$ is linear
in $\mu_1$. Or, $\mathbb{E}\left[X - \mu_1 | X > \mu_1 \right]$ is
simply the mean of excesses above the threshold $\mu_1$ which can
easily be estimated using the empirical mean.

A mean residual life plot consists in representing points:
\begin{equation}
  \label{eq:mrlplot}
  \left\{\left(\mu, \frac{1}{n_\mu} \sum_{i=1}^{n_\mu} x_{i, n_\mu} -
      \mu \right) : \mu \leq x_\mathrm{max} \right\}
\end{equation}
where $n_\mu$ is the number of observations $\mathbf{x}$ above the
threshold $\mu$, $x_{i, n_\mu}$ is the $i$-th observation above the
threshold $\mu$ and $x_\mathrm{max}$ is the maximum of the
observations $\mathbf{x}$.

Confidence intervals can be added to this plot as the empirical mean
can be supposed to be normally distributed (Central Limit
Theorem). However, normality doesn't hold anymore for high threshold
as there are less and less excesses. Moreover, by construction, this
plot always converge to the point $(x_\mathrm{max}, 0)$.

Here is another synthetic example.
\begin{verbatim}
> x <- rnorm(10000)
 mrlplot(x, u.range = c(1, 3.5), col = c("green", "black", "green"))
\end{verbatim}

Figure~\ref{fig:mrlplot} displays the mean residual life plot. A
threshold around 2.5 should be reasonable.
\begin{figure}
  \centering
  \includegraphics[angle=-90,width=\textwidth]{mrlplot}
  \caption{Mean residual life plot on synthetic data}
  \label{fig:mrlplot}
\end{figure}

\subsubsection{L-Moments plot: \emph{lmomplot}}

L-moments are summary statistics for probability distributions and
data samples. They are analogous to ordinary moments -- they provide
measures of location, dispersion, skewness, kurtosis, and other
aspects of the shape of probability distributions or data samples --
but are computed from linear combinations of the ordered data values
(hence the prefix L).

For the GPD, the following relation holds:
\begin{equation}
  \label{eq:lomRel}
  \tau_4 = \tau_3 \frac{1 + 5 \tau_3}{5 + \tau_3}
\end{equation}
where $\tau_4$ is the \textbf{L-Kurtosis} and $\tau_3$ is the
\textbf{L-Skewness}.

The \textbf{L-Moment} plot represents points defined by:
\begin{equation}
  \label{eq:lmomplot}
  \left\{\left(\hat{\tau}_{3,u}, \hat{\tau}_{4,u}\right) : u \leq
    x_\mathrm{max} \right\}  
\end{equation}
where $\hat{\tau}_{3,u}$ and $\hat{\tau}_{4,u}$ are estimations of the
L-Kurtosis and L-Skewness based on excesses over threshold $u$ and
$x_\mathrm{max}$ is the maximum of the observations $\mathbf{x}$. The
theoretical curve defined by equation~\eqref{eq:lomRel} is traced as a
guideline.

Here is a trivial example.
\begin{verbatim}
> x <- c(1 - abs(rnorm(200, 0, 0.2)), rgpd(100, 1, 2, 0.25))
> lmomplot(x, u.range = c(0.9, 2), identify = FALSE)
\end{verbatim}

Figure~\ref{fig:lmomplot} displays the L-Moment plot. By passing
option \verb+identiy = TRUE+ user can click on the graphic to identify
the threshold related to the point selected.

\begin{figure}
  \centering
  \includegraphics[angle=-90,width=\textwidth]{lmomplot}
  \caption{L-Moment plot on synthetic data}
  \label{fig:lmomplot}
\end{figure}

We found that this graphic has often poor performance on real data.

\subsubsection{Dispersion Index Plot: \emph{diplot}}

The \textbf{Dispersion Index plot} is particularly useful when dealing
with time series. The EVT states that excesses over a threshold can be
approximated by a GPD. However, the EVT also states that the
occurrences of these excesses must be represented by a Poisson
process.

Let $X$ be a \textit{r.v.} distributed as a Poisson distribution with
parameter $\lambda$. That is:
\begin{equation}
  \label{eq:poissLaw}
  \Pr\left[X = k\right] = e^{-\lambda} \frac{\lambda^k}{k!}, \quad k
  \in \mathbb{N}.
\end{equation}
Thus, we have $\mathbb{E}\left[X\right] =
Var\left[X\right]$. \citet{Cunnane1979} introduced a
\textbf{Dispersion Index} statistic defined by:
\begin{equation}
  \label{eq:DI}
  DI = \frac{s^2}{\lambda}
\end{equation}
where $s^2$ is the intensity of the Poisson process and $\lambda$ the
mean number of events in a block - most often this is a
year. Moreover, a confidence interval can be computed by using a
$\chi^2$ test:
\begin{equation}
  \label{eq:confDI}
   I_{\alpha} = \left[ \frac{\chi^2_{\left(1 - \alpha\right) / 2,
        M-1}}{M-1}, \frac{\chi^2_{1 - \left( 1 -\alpha\right) / 2,
        M-1}}{M-1} \right]
\end{equation}
where $\Pr\left[ DI \in I_{\alpha} \right] = \alpha$.

For the next example, we use the data set \emph{ardieres} included in
the \textbf{POT} package. Moreover, as \emph{ardieres} is a time
series, and thus strongly auto-correlated, we must ``extract'' extreme
events while preserving independence between events. This is achieved
using function \textbf{clust}\footnote{The clust function will be
  presented later in section~\ref{subsec:declust}.}.
\begin{verbatim}
> data(ardieres)
> events <- clust(ardieres, u = 2, tim.cond = 8 / 365,
+ clust.max = TRUE)
> diplot(events, u.range = c(2, 20))
\end{verbatim}

The Dispersion Index plot is presented in
Figure~\ref{fig:diplot}. From this figure, a threshold around 5 should
be reasonable.

\begin{figure}
  \centering
  \includegraphics[angle=-90,width=\textwidth]{diplot}
  \caption{Dispersion index plot for the dataset \emph{ardieres}}
  \label{fig:diplot}
\end{figure}

\subsection{Fitting the GPD}
\label{subsec:fitGPD}

The main function to fit the GPD is called \textbf{fitgpd}. This is a
generic function which can fit the GPD according several
estimators. There are currently 7 estimators available: method of
moments \verb|moments|, maximum likelihood \verb|mle|, biased and
unbiased probability weighted moments \verb|pwmb, pwmu|, mean power
density divergence \verb|mdpd|, median \verb|med| and pickands'
\verb|pickands| estimators. Details for these estimators can be found
in~\citep{Coles2001}, \citep{Hosking1987}, \citep{Juarez2004},
\citep{Peng2001} and \citep{Pickands1975}.

The MLE is a particular case as it is the only one which allows
varying threshold. Moreover, two types of standard errors are
available: ``expected'' or ``observed'' information of Fisher. The
option \verb|obs.fish| specifies if we want observed
(\verb|obs.fish = TRUE|) or expected (\verb|obs.fish = FALSE|).

As Pickands' estimator is not always feasible, user must check the
message of feasibility return by function \verb+fitgpd+.

We give here several didactic examples.
\begin{verbatim}
> x <- rgpd(200, 1, 2, 0.25)                             
> mom <- fitgpd(x, 1, "moments")$param                        
> mle <- fitgpd(x, 1, "mle")$param                       
> pwmu <- fitgpd(x, 1, "pwmu")$param                     
> pwmb <- fitgpd(x, 1, "pwmb")$param                     
> pickands <- fitgpd(x, 1, "pickands")$param             
> med <- fitgpd(x, 1, "med", start = mle)$param          
> mdpd <- fitgpd(x, 1, "mdpd")$param                     
> print(rbind(mom, mle, pwmu, pwmb, pickands, med, mdpd))
            scale      shape
mom      1.693222 0.22258795
mle      1.758304 0.18825359
pwmu     1.774835 0.18511688
pwmb     1.783859 0.18097373
pickands 1.867132 0.02210745
med      1.826378 0.08445534 ##Convergence: iteration limit reached
mdpd     1.788432 0.16608265
\end{verbatim}

The MLE method allows to fix either the scale or the shape
parameter. For example, if we want to fit a Exponential distribution,
just do:
\begin{verbatim}
> x <- rgpd(100, 1, 2, 0)
> fitgpd(x, thresh = 1, shape = 0, method = "mle")
> ##The same but with a fixed scale value
> fitgpd(x, thresh = 1, scale = 2, method = "mle")
\end{verbatim}
If now, we want to fit a GPD with a varying threshold, just do:
\begin{verbatim}
> x <- rgpd(500, 1:2, 0.3, 0.01)
> fitgpd(x, 1:2, method = "mle")
\end{verbatim}

Note that the varying threshold is repeated cyclically until it
matches the length of object \verb|x|.
 
\subsection{Confidence Intervals}
\label{subsec:confInt}

Once a statistical model is fitted, it is usual to gives confidence
intervals. Currently, only \verb|mle, pwmu, pwmb, moments| estimators
can computed confidence intervals. Moreover, for method \verb|mle|,
``standard'' and ``profile'' confidence intervals are available.

If we want confidence intervals for the scale parameters:
\begin{verbatim}
> x <- rgpd(100, 1, 2, 0.25)
> mle <- fitgpd(x, 1, method = "mle")
> mom <- fitgpd(x, 1, method = "moments")
> pwmb <- fitgpd(x, 1, method = "pwmb")
> pwmu <- fitgpd(x, 1, method = "pwmu")
> gpd.fiscale(mle, conf = 0.9)
> gpd.fiscale(mom, conf = 0.9)
> gpd.fiscale(pwmu, conf = 0.9)
> gpd.fiscale(pwmb, conf = 0.9)
\end{verbatim}
For shape parameter confidence intervals, simply use function
\verb|gpd.fishape| instead of \verb|gpd.fiscale|. Note that the
\emph{fi} stands for ``Fisher Information''.

Thus, if we want profile confidence intervals, we must use functions
\verb|gpd.pfscale| and \verb|gpd.pfshape|. The \emph{pf} stands for
``profile''. These functions are only available with a model fitted
with MLE.
\begin{verbatim}
> gpd.pfscale(mle, range = c(1, 2.5), conf = 0.9)
> gpd.pfshape(mle, range = c(-0.1, 0.6), conf = 0.95)
\end{verbatim}

Confidence interval for quantiles - or return levels - are also
available. This is achieved using: (a) the Delta method or (b) profile
likelihood.
\begin{verbatim}
> gpd.firl(pwmu, prob = 0.95)
> gpd.pfrl(mle, prob = 0.95, range = c(4.8, 15))
\end{verbatim}

The profile confidence interval functions both returns the confidence
interval and plot the profile log-likelihood
function. Figure~\ref{fig:pfrl} depicts the graphic window returned by
function \verb|gpd.pfrl| for the return level associated to non
exceedence probability 0.95.
\begin{figure}
  \centering
  \includegraphics[angle=-90,width=\textwidth]{pfrl}
  \caption{Profile log-likelihood function for a given return level}
  \label{fig:pfrl}
\end{figure}

\subsection{Model Checking}
\label{subsec:modCheck}


To check the fitted model, users must call function
\textbf{plotgpd}. This is a generic function which calls functions:
\verb|pp.gpd| (probability/probability plot), \verb|qq.gpd|
(quantile/quantile plot), \verb|dens.gpd| (density plot) and
\verb|retlev.gpd| (return level plot).

Here is a basic illustration of the function \verb|plotgpd|.
\begin{verbatim}
> x <- rgpd(200, 10, 0.5, -0.2)
> fitted <- fitgpd(x, 10, method = "mle")
> par(mfrow=c(2,2))
> plotgpd(fitted, npy = 1)
\end{verbatim}

\begin{figure}
  \centering
  \includegraphics[angle=-90,width=\textwidth]{plotgpd}
  \caption{Checking plots from function \emph{plotgpd}}
  \label{fig:plotgpd}
\end{figure}

Figure~\ref{fig:plotgpd} displays the graphic windows obtained with
the latter execution.

If one is interested in only a probability/probability plot, there is
two options. We can call function \verb|pp.gpd| or equivalently
\verb|plotgpd| with the \textbf{which} option. The ``which'' option
select which graph you want to plot. That is:
\begin{itemize}
\item{which = 1} for a probability/probability plot;
\item{which = 2} for a quantile/quantile plot;
\item{which = 3} for a density plot;
\item{which = 4} for a return level plot;
\end{itemize}
Note that ``which'' can be a vector like \verb|c(1,3)| or \verb|1:3|.

Thus, the following instruction gives the same graphic.
\begin{verbatim}
> plotgpd(fitted, which = 1)
> pp.gpd(fitted)
\end{verbatim}

If a return level plot is asked ($4 \in$ \verb|which|), a value for
\verb|npy| is needed. ``npy'' corresponds to the \emph{mean number of
  events per year}. This is required to define the ``return period''.

\subsection{Declustering Techniques}
\label{subsec:declust}

In opposition to block maxima, a peak over threshold can be
problematic when dealing with time series. Indeed, as often time
series are strongly auto-correlated, select naively events above a
threshold may lead to dependent events.

The function \textbf{clust} tries to identify peaks over a threshold
while meeting independence criteria. For this purpose, this function
needs at least two arguments: the threshold \verb|u| and a time
condition for independence \verb|tim.cond|. Clusters are identify as
follow:
\begin{enumerate}
\item The first exceedence initiates the first cluster;
\item The first observation under the threshold \verb|u| ``ends'' the
  cluster unless \verb|tim.cond| does not hold;
\item The next exceedence which hold \verb|tim.cond| initiates a new
  cluster;
\item The process is iterated as needed.
\end{enumerate}

Here is an application on flood discharges for river Ardi\`ere at
Beaujeu. A preliminary study shows that two flood events can be
considered independent if they do not lie within a 8 days window. Note
that unit to define \verb|tim.cond| must be the same than the data
analyzed. 
\begin{verbatim}
> data(ardieres)
> clust(ardieres, u = 2, tim.cond = 8 / 365)
\end{verbatim}

Several options can be passed to the ``clust'' function. By default,
it will return a list with the identified clusters. Usually, we want
only cluster maxima, this is achieved by passing option
\verb|clust.max = TRUE|. Users can also ask for a graphic
representation of clusters by passing option \verb|plot = TRUE| - see
Figure~\ref{fig:clust}.
\begin{verbatim}
> clustMax <- clust(ardieres, u = 2, tim.cond = 8 / 365,
+ clust.max = TRUE, plot = TRUE, xlim = c(1971.1, 1972.9))
\end{verbatim}
\begin{figure}
  \centering
  \includegraphics[angle=-90,width=\textwidth]{clust}
  \caption{The identified clusters. Data Ardi\`eres, u = 2, tim.cond = 8}
  \label{fig:clust}
\end{figure}

\subsection{Miscellaneous functions}
\label{subsec:miscFunc}

\subsubsection{Return periods: \emph{rp2prob} and \emph{prob2rp}}

The functions \textbf{rp2prob} and \textbf{prob2rp} are useful to
convert return periods to non exceedence probabilities and vice
versa. It needs either a return period either a non exceedence
probability. Moreover, the mean number of events per year ``npy'' must
be specified.
\begin{verbatim}
> rp2prob(50, 1.8)
  npy retper      prob
1 1.8     50 0.9888889
> prob2rp(0.6, 2.2)
  npy   retper prob
1 2.2 1.136364  0.6
\end{verbatim}

\subsubsection{Unbiased Sample L-Moments: \emph{samlmu}}

The function \textbf{samlmu} computes the unbiased sample L-Moments.
\begin{verbatim}
> x <- runif(50)
> samlmu(x, nmom = 5)
        l_1         l_2         t_3         t_4         t_5 
 0.53337554  0.16743489 -0.04026843  0.01243610  0.01386457
\end{verbatim}

\subsubsection{Mobile average window on time series: \emph{ts2tsd}}

The function \textbf{ts2tsd} computes an ``average'' time series
\verb|tsd| from the initial time series \verb|ts|. This is achieved by
using a mobile average window of length \verb|d| on the initial time
series.
\begin{verbatim}
> data(ardieres)
> tsd <- ts2tsd(ardieres, 3 / 365)
> plot(ardieres, type = "l", col = "blue")
> lines(tsd, col = "green")
\end{verbatim}
The latter execution is depicted in Figure~\ref{fig:ts2tsd}.
\begin{figure}
  \centering
  \includegraphics[angle=-90,width=\textwidth]{ts2tsd}
  \caption{Instantaneous flood discharges and averaged dischaged over
    duration 3 days. Data ardieres}
  \label{fig:ts2tsd}
\end{figure}

\section{A Concrete Statistical Analysis of Peaks Over a Threshold}
\label{sec:concAn}

In this section, we provide a full and detailed analysis of peaks over
a threshold for the river Ardi\`eres at
Beaujeu. Figure~\ref{fig:ts2tsd} depicts instantaneous flood
discharges - blue line.

As this is a time series, we must selects independent events above a
threshold. First, we fix a relatively low threshold to ``extract''
more events. Thus, some of them are not extreme but regular
events. This is necessary to select a reasonable threshold for the
asymptotic approximation by a GPD - see section~\ref{sec:introEVT}.
\begin{verbatim}
> summary(ardieres)
      time           obs        
 Min.   :1970   Min.   : 0.022  
 1st Qu.:1981   1st Qu.: 0.236  
 Median :1991   Median : 0.542  
 Mean   :1989   Mean   : 1.024  
 3rd Qu.:1997   3rd Qu.: 1.230  
 Max.   :2004   Max.   :44.200  
                NA's   : 1.000  
> events0 <- clust(ardieres, u = 1.5, tim.cond = 8/365,
+  clust.max = TRUE)
> par(mfrow=c(2,2))
> mrlplot(events0[,"obs"])
> abline( v = 6, col = "green")
> diplot(events0)
> abline( v = 6, col = "green")
> tcplot(events0[,"obs"])
\end{verbatim}
From Figure~\ref{fig:threshSelect}, a threshold value of $6 m^3/s$
should be reasonable. The Mean residual life plot - top left panel-
indicates that a threshold around $10 m^3/s$ should be
adequate. However, the selected threshold must be low enough to have
enough events above it to reduce variance while not too low as it
increase the bias\footnote{As the asymptotic approximation by a GPD is
  not accurate anymore.}.

\begin{figure}
  \centering
  \includegraphics[angle=-90,width=\textwidth]{threshSelect}
  \caption{Threshold selection for river Ardi\`eres at Beaujeu.}
  \label{fig:threshSelect}
\end{figure}

Thus, we can now ``re-extract'' events above the threshold $6 m^3/s$,
obtaining object \verb|events1|. This is necessary as sometimes
\verb|events1| is not equal to observations of \verb|events0| greater
than $6 m^3/s$. We can now define the mean number of events per year
``npy''. Note that an estimation of the extremal index is available.
\begin{verbatim}
> events1 <- clust(ardieres, u = 6, tim.cond = 8/365,
+ clust.max = TRUE)
> npy <- length(events1[,"obs"]) / (diff(range(ardieres[,"time"],
 na.rm = TRUE)) - diff(ardieres[c(20945,20947),"time"]))
> ##Because there is a gap !!!
> print(npy)
[1] 1.677934
> attributes(events1)$exi
[1] 0.1225383
\end{verbatim}

Let's fit the GPD.
\begin{verbatim}
> mle <- fitgpd(events1[,"obs"], thresh = 6, method = "mle")
Estimator: MLE 

Varying Threshold: FALSE 

Threshold: 6 
Number Above: 56 
Proportion Above: 1 

Estimates
 scale   shape  
3.8285  0.1579  

Standard Error Type: Observed 

Standard Errors
 scale   shape  
0.7224  0.1349  

Asymptotic Variance Covariance
       scale     shape   
scale   0.52180  -0.05714
shape  -0.05714   0.01819

Optimization Information
  Convergence: successful 
  Function Evaluations: 36 
  Gradient Evaluations: 9 
\end{verbatim}
The result of function \textbf{fitgpd} gives the name of the
estimator, if a varying threshold was used, the threshold value, the
number and the proportion of observations above the threshold,
parameter estimates, standard error estimates and type, the asymptotic
variance-covariance matrix and convergence diagnostic.

\begin{figure}
  \centering
  \includegraphics[angle=-90,width=\textwidth]{checkArdieres}
  \caption{Graphic diagnostics for river Ardi\`eres at Beaujeu}
  \label{fig:checkArdieres}
\end{figure}

Figure~\ref{fig:checkArdieres} shows graphic diagnostics for the
fitted model. It can be seen that the fitted model ``mle'' seems to be
appropriate. Suppose we want to know the return level associated to
the 100-year return period.
\begin{verbatim}
> ##First convert return period in prob
> rp2prob(retper = 100, npy = npy)
       npy retper      prob
1 1.677934    100 0.9940403
> prob <- rp2prob(retper = 100, npy = npy)[,"prob"]
> qgpd(prob, loc = 6, scale = mle$param["scale"],
+  shape = mle$param["shape"])
36.19317 
\end{verbatim}


To take into account uncertainties, Figure~\ref{fig:pfrlArdieres}
depicts the profile confidence interval for the quantile associated to
the 100-year return period.
\begin{verbatim}
> gpd.pfrl(mle, prob, range = c(25, 90), nrang = 200)
If there is some troubles try to put vert.lines = FALSE or change
 the range...
conf.inf conf.sup 
25.48995 87.87688 
\end{verbatim}
\begin{figure}
  \centering
  \includegraphics[angle=-90,width=\textwidth]{pfrlArdieres}
  \caption{Profile-likelihood function for the 100-year return period
    quantile} 
  \label{fig:pfrlArdieres}
\end{figure}

Sometimes it is necessary to know the estimated return period of a
specified events. Lets do it with the larger events in ``events1''.
\begin{verbatim}
> maxEvent <- max(events1[,"obs"])
> print(maxEvent)
[1] 44.2
> prob <- pgpd(maxEvent, loc = 6, scale = mle$param["scale"],
+ shape = mle$param["shape"])
> print(prob)
0.997501 
> prob2rp(prob, npy = npy)
       npy   retper     prob
1 1.677934 238.4804 0.997501
\end{verbatim}
Thus, the largest events that occurs in June 2000 has approximately a
return period of 240 years.

Maybe it is a good idea to fit the GPD with the other estimators
available in the \textbf{POT} package.
\bibliography{biblio}
\end{document}
%%% Local Variables: 
%%% mode: latex
%%% TeX-master: t
%%% End: 
